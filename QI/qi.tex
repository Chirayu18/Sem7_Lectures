\documentclass{scrartcl}
\usepackage[utf8]{inputenc}
\usepackage{physics}
\usepackage[
backend=biber,
style=numeric,
sorting=ynt
]{biblatex}
\usepackage{amsmath}
\usepackage{graphicx}
\newtheorem{theorem}{Theorem}
\usepackage[a4paper, total={6in, 9in}]{geometry}
\addbibresource{references.bib}
\title{  \\ \large{ Course Code } }

\author{Chirayu Gupta}
\date{}
\begin{document}
\section{Lecture 1}
\begin{itemize}
	\item If you can compute anything then a turing machine must be able to solve it.
	\item Polynomial time: Order is \( \order{n^\alpha} \)
	\item Quantum computers claim to solve non-deterministic polynomial(NP) problems in polynomial time
\end{itemize}
\begin{theorem}
	\textbf{Moore's Law}: The number of transistors that you can put in an IC doubles every 2 years.
\end{theorem}
\begin{itemize}
	\item Cant pack transistors more and more because at small transistors diameters quantum tunneling can take place
	\item "Lower Limit"
	\item Hence Quantum computers can help
\end{itemize}
\subsection{Basics of Quantum Physics}
\( \psi \) denotes state of a system: I can compute anything with \textbf{only} this.
In position basis, I can have
\[ \bra{x}\ket{\psi} = \psi(x) \]
Or,
\[ \bra{p}\ket{\psi} = \psi(p) \]
Reside in "Hilbert Space"(A linear vector space with an inner product, Infinite dimension)
Can be finite dimension hilbert space
\[ L^2 \rightarrow l(l+1)h \]\[ L_z \rightarrow mh (-l\leq m \leq l) \]
Operator is an abstract object until you choose a representation.
Applying a operator is same as applying a matrix multiplication.
In a eigenbasis for an operator:
\[ \hat{O} \ket{\psi} = a \ket{\psi} \]
\( \ket{\psi} \) only increases or decreases in length\\
Quantum computing is all about operators on states
\[ O \ket{\psi} = \ket{\psi'} \]
\begin{equation}
	\ket{\psi} = \displaystyle\sum_{i=1}^{M}a_i \ket{\phi_i}
	\label{eq:cmp}
\end{equation}
We will be dealing with TDSE:
\begin{equation}
	ih \pdv{\psi(x,t)}{t} = \hat{H}\psi(x,t)
	\label{eq:tdse}
\end{equation}
\textbf{Time Evolution}
\begin{equation}
	\psi(x,t) = e^{-\iota\hat{H}t/h} \psi(x,0)
	\label{eq:timeevol}
\end{equation}
This is the Scrodinger picture, it is for non-time dependent hamiltonian
\[ \psi(x,t) = \hat(U) \psi(x,0) \]
Where U is unitary operator\\
If:
\[ H \rightarrow H(t) \]
\[ U(t_0,t) = \mathbf{T} e^{-\iota/h \displaystyle\int_{t_0}^{t}dt' H(t')} \]
It can be shown that:
\[ U^\dagger U = I \]
Also:\\
Eigenvalues of U are of modulo 1 (TODO:Prove this)
\begin{theorem}
	For any hermitian Matrix A
	\begin{displaymath}
		e^{\iota A} = \hat{U}
	\end{displaymath}
	Where U is unitary operator	
\end{theorem}
For 2 non commuting operators A and B
\[ e^{A + B)} \neq e^A \cdot e^B \]
For \( [ [A,B], A] =0 \) (TODO:See proof)
\[ e^{A + B} = e^A e^B e^{-[A,B]/2} \]
\[ A \ket{\psi} = a \ket{\psi} \]\[ A \ket{\psi} \bra{\psi} \ket{\psi} = a \ket{\psi} \]
Replace \(  \ket{\psi} \bra{\psi}\) by \[ \hat{\rho} \]
which is the density operator
\[\bra{\psi} \rho \ket{\psi} = \abs{\psi}^2 \]
\( \rho \) is as important as \( \psi \). Real importance is apparent in ensemble of systems.\\
\textbf{Ensemble} is a collection of identical systems. Example a collection of \( H_2 \) atoms in different states. No use if all are in the same state.\\
Suppose I have N states with \( N_1 \) of them in state \( \psi_1 \), \( N_2 \) in \( \psi_2 \) etc.
So instead of just \( \psi \), I have and ensemble of states: "Mixed states" 
\begin{equation}
\begin{bmatrix}
	 N_1 \psi_1\\
	 N_2 \psi_2\\
	 N_3 \psi_3\\
	...
\end{bmatrix}
\end{equation}
\end{document}
