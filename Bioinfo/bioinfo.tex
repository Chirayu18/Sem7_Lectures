\documentclass{scrartcl}
\usepackage[utf8]{inputenc}
\usepackage[
backend=biber,
style=numeric,
sorting=ynt
]{biblatex}
\usepackage{amsmath}
\usepackage{graphicx}
\newtheorem{theorem}{Theorem}
\usepackage[a4paper, total={6in, 9in}]{geometry}
\addbibresource{references.bib}
\title{ BioInformatics \\ \large{ Course Code } }

\author{Chirayu Gupta}
\date{}
\begin{document}
\maketitle
\section{PAM Paper}
\begin{itemize}
	\item Relationship between sequences are important to study evolutionary parameters
	\item Accepted point mutation is a replacement of one amino acid with another accepted by natural selection
	\item There are 20 amino acids in nature(I think)
	\item Replacement of X by Y is same as Y by X
	\item Non transitive nature of accepted point mutations
	\item  Pam matices value \(  M_{ij} = \displaystyle\frac{\lambda m_j A_{ij}}{\displaystyle\sum_{i}^{}A_{ij}} \)
	\item \( \lambda \) is prop contant. \( A_{ij} \) is element of the point mutation matix \( m_j \) is the mutability of jth amino acid
	\item Total must sum to 1 so diagonal elements are \( M_{jj} = 1 - \lambda m_j \)
	\item PAM decribes a unit of evolutionary change
	\item Applying PAM matrix N times gives matix after N such evolutionary changes
	\item \( f_i \) is the probablity that i will occur in the second sequence by chance(I think its the sum of elements in the corresponding row along with some normalization)
	\item Percentage of amino acids observed to change on average \( 100(1 - \displaystyle\sum_{i}^{}f_i M_{ii}) \)
	\item Relatedness Odds Matrix \( R_{ij} = \displaystyle\frac{M_{ij}}{f_i} \) is a matrix giving probablity of replacement per occurrence of i per occurrence of j
	\item PAM 250 is relatedness odds matrix. In log scale
\end{itemize}
\section{Dynamic Programming Overview}
\textbf{Why need an algorithm?}\\
Complexity of scoring all possible sequences is \( O(2^{2N}) \).\\

\end{document}
