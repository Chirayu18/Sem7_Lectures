\documentclass{scrartcl}
\usepackage[utf8]{inputenc}
\usepackage{physics}
\usepackage[
backend=biber,
style=numeric,
sorting=ynt
]{biblatex}
\usepackage{amsmath}
\usepackage{graphicx}
\newtheorem{theorem}{Theorem}
\usepackage[a4paper, total={6in, 9in}]{geometry}
\addbibresource{references.bib}
\title{ Quantum Field Theory \\ \large{ PH } }
\author{Chirayu Gupta}
\date{}
\begin{document}
\maketitle
\section{Lecture 1}
All states contribute to relativistic corrections:
\[ H \rightarrow H + \partial V \]
Where the corrections in energy is: 
\[ E_{corr} = \bra{0} \partial V \ket{0} + \sum_n\displaystyle\frac{\abs{\bra{0}\partial V \ket{n}}^2}{E_0 - E_n}  + ... \]
As you can see it depends on n.\\
Since for relativistic cases processes like 
\[ p + p \rightarrow p + p + \pi_0 \] are possible. That is new particles can be created,
the number of such states \( \ket{n} \) are infinite.\\
Typically, the relativistic corrections are af the order \( \order{v^2} \) because, typically the denominator
is \( mc^2 \) and since \[ \displaystyle\frac{E}{mc^2} = \displaystyle\frac{\gamma c^2}{c^2} = \displaystyle\frac{1}{\sqrt{1-v^2/c^2}} = 1+ \displaystyle\frac{v^2}{c^2} \]
\section{Lecture 2}
\subsection{Natural Units}
\begin{itemize}
	\item \( h_{cross} = 1 \)
	\item \( c = 1 \)
\end{itemize}
Above 2 conditions are for natural units.
There are implications on dimentions because of this which are:
\begin{itemize}
	\item \( [m] = [E] = [L^{-1}] = [T^{-1}]  \)
\end{itemize}
\subsection{Theory of Single free spinless particle of mass \( \mu \)[From coleman notes]}
Momentum operator:
\[ \va{P} \ket{k} = k \ket{k} \]
\[ \bra{k} \ket{k'} = \partial (k-k') \]
Completeness:
\( \int_{}^{} \bra{k} \ket{k} d^3 k = 1 \)\\
We know:
\[ \psi(k) = \bra{k}\ket{\psi} \]
Also classically we know:
\[ H \ket{k} = \abs{k}^2 / 2 \mu \ket{k} \]
To introduce relativity we introduce:
\[ H \ket{k} = \sqrt{\abs{k}^2  + \mu^2} \cdot \ket{k} = \omega \ket{k} \]
How do we know this makes the theory relativistic(or Lorentz invariant)?\\
First we define Translational and Rotational Invariance:
\subsection{Translational Invariance}
For a linear operator\( U \) specifying a translation: we have
\begin{equation}
	U(a)U(a)^\dagger = 1
	\label{eq:ti_1}
\end{equation}
\begin{equation}
	U(0) = 1
	\label{eq:ti_2}
\end{equation}
\begin{equation}
	U(a)U(b) = U(a+b)
	\label{eq:ti_3}
\end{equation}
The U satisfying these is \( U(a) = e^{\iota \mathbf{p} a} \) where \( \mathbf{p} = (H,\va{p}) \) \\
\[ U(a)\ket{0} = \ket{a} \]
\[ \bra{a}O(x+a)\ket{a} = \bra{0}O(x)\ket{0} \]
\subsection{Rotational Invariance}
Similar to [Equation \ref{eq:ti_1} - \ref{eq:ti_3}], we have equations for rotation
\begin{equation}
	U(R)U(R)^\dagger = 1
	\label{eq:ri_1}
\end{equation}
\begin{equation}
	U(1) = 1
	\label{eq:ri_2}
\end{equation}
\begin{equation}
	U(R_1)U(R_2) = U(R_1 R_2)
	\label{eq:ri_3}
\end{equation}
A U satisfying all these is given by 
\[ U(R)\ket{\va{k}} = \ket{R \va{k}} \]
TODO: Put Lorentz invariance and the relativistic contour integral calculations from coleman
\section{Lecture 3}
\subsection{Klien-Gordan equation}
A heuristic way to derive SC equation:
\[ E = \displaystyle\frac{P^2}{2m} \]
\[ E \rightarrow \iota \pdv{}{t} \]
\[ p_j \rightarrow - \iota \pdv{}{x_j} \]
Action is defined as:
\[ \mathbf{S}[\phi] = \displaystyle\int_{}^{}(\partial^\mu \phi \partial_\mu \phi - m^2 \phi^2(x)) \]
Similarly we derive KG eqn:
\[ E^2 = p^2 + m^2 \]
\[ (\pdv{}{t} - \nabla^2)\phi + m^2 \phi(x,t) = 0 \]
A solution to this equation is \( e^{\iota(k \cdot x + \omega t)} \). When you apply \( \hat{H} \) we get energies like \( -\omega \phi \).\textbf{Negative Energy solution}. If you perturb this system slightly it keeps going down the potential well(since energy not bounded from below).  
In QM continuity eqn:
\[ \nabla J + \pdv{\rho}{t} = 0 \]
\[J_i = \frac{ 1 }{ 2im } ( \phi^* \partial \phi - (\partial \phi^*) \phi)\]
\[ \rho = \displaystyle\frac{\iota}{2m}(\phi^* \pdv{\phi}{t} - \pdv{\phi^*}{t}) \]
With KG its same but \( \rho \) is not positive definite.\\
Hence, KG equation failed!.\\
TODO: Dont really understand why KG failed
\subsection{Rotation}
Rotation is like a gauge transformation: we apply some transformation on \( \hat{A}, \hat{\phi} \) which does not change E and B. The choice of coordinations/description has nothing to do with the system!!\\
Quantites that don't change under a transformation are \textbf{scalar} under that transformation.
Eg: \( x^2 + y^2 \) does not depend on choice of axes given a origin.\\
A scalar is always associated with a transformation.
\subsection{Scalar Field, Vector Field}
Given a point and density of air(\( \rho \)) we take two coordinate systems or 2 different description:
\[ \rho(\va{x}) \rightarrow \rho(R \va{x}) \]
Rho is a scalar field.\\
For a vector field under rotation:
\[ E_x(\va{X},\va{Y}) , E_y(\va{X},\va{Y}) \rightarrow R E(R \va{X}) \]
Both field components and arguments changed.\\
\textbf{Question}: I do a rotation on the coordinates then what is the effent on components of \( \psi_n(\va{X}) \)\\
\[ \mathbf{R} \psi_{ij}(R \va{x}) \]
\( \mathbf{R} \) is representation of the Rotation group \( SO(3) \)
\subsection{Lorentz Group}
Group is \[ SO(1,3)\] not \( SO(4) \)
in which \[ (-x_0^2 + \displaystyle\sum_{}^{} x_i^2) \]
is invariant.\\
Transformation is represented by:
\[ \mathbf{R}_{ij} \psi_j ( \Lambda \va{x}) \]
\begin{equation}
	\eta = Diag(-1,1,1,1)
	\label{eq:mmatrix}
\end{equation}
Lambda satisfies:
\[ \Lambda^t \eta \Lambda = \eta \]
\[ \Lambda_0^0 \geq 1 \rightarrow Orthochoronum \]
\[ \Lambda_0^0 \leq 1 \rightarrow Not\ Considered\ as\ impllies\ time\ reversal \]
Formally Lorentz group defined as \( SO^+ ( 1,3) \) to denote orthochronum.\\
For \( SO(1,3) \) \( Det \Lambda = 1 \)
For parity: \( \Lambda = diag(-1,1,1,1) \)\\
For time reversal: \( \Lambda = diag(1,-1,-1,-1) \)\\
Parity and time reversal are outside the \( SO(1,3) \) group.\\
For 2-D parity is inside \( SO(1,3) \) (TODO:Why?)

\section{Lecture 4}
\subsection{Classical Field Theory}
\[ \partial \mathbf{S} = \partial \displaystyle\int_{}^{} dt L(x,\dot{x}) = 0\]
Subject to constraint:
\[ \partial x |_{t_1,t_2} = 0 \]
For QFT:
\[ \mathbf{S} = \displaystyle\int_{}^{} d^4 x \mathbf{L}(\phi(x),\partial_\mu \phi(x)) \]
\[ \partial \mathbf{S} = 0 \]
Boundary condition:
\[ \partial \phi(x) = 0 \]
\[ \delta S = \displaystyle\int_{}^{}d^4 x [ \displaystyle\pdv{L}{\phi(x)}] \delta \phi(x) + \displaystyle\pdv{L}{\partial_\mu \phi(x)} \partial(\delta \phi(x)) \]
\[ \delta S = \displaystyle\int_{}^{}d^4 x [ \displaystyle\pdv{L}{\phi(x)}] \delta \phi(x) + - \partial_\mu \displaystyle\pdv{L}{\partial_\mu \phi(x)} \delta \phi(x) + \partial_\mu \displaystyle\pdv{L}{\partial_\mu \phi(x)} \delta \phi\]
On solving we get:
\[ \pdv{L}{\phi(x)} - \partial_\mu \displaystyle\frac{\partial L}{\partial(\partial_\mu \phi(x))} = 0 \]
\begin{itemize}
	\item In a Boundary condition: we set \( \phi \) at $t_1$ and $t_2$. For a double derivative equation: we can have only 2 boundary condition hence cant set velocity at \( t_1 \) and \( t_2  \) as well
	\item Lagrangian is taken to be real
	\item If Lagrangian not real Hamiltonian also not real\( \rightarrow \) problem in time evolution
	\item Lagrangian is Lorentz invariant
\end{itemize}
\[ L = (\partial_\mu \phi)(\partial^\mu \phi) - m^2 \phi^2 \]
\[ \pdv{L}{\phi} = -2m^2 \phi \]
\[ \partial_\mu \pdv{L}{\partial_\mu \phi} = 4 \partial_\mu (\partial^\mu \phi) \]
This gives KG equation(TODO:How?)
\[ (\partial_\mu \partial^\mu + m^2 ) \phi = 0 \]
If we write a new Lagrangian: 
\[ L' = - \phi \partial_\mu \partial^\mu - m^2 \phi^2 \]
\[  = L - \partial_\mu ( \phi \partial^\mu \phi) \]
On putting in action principle we get an extra \( \partial \phi \) (which can be cancelled as we can change that in BT)  and \(   \partial_\mu ( \phi \partial^\mu \phi) \) which need to be cancelled \textbf{hence in defining Lagrangian like this we might need to add an extra boundary terms} which can be important in many theories.
\[ S = \displaystyle\int_{}^{}d^4 L + BT \]
\subsection{Global Symeetries(Reference Wienberg: Section 7.3)}
\[ \psi(x) \rightarrow \psi(x) + \iota \epsilon F(x)\]
\[ \delta S = \iota \epsilon \displaystyle\int_{\displaystyle\frac{\delta S[\psi]}{\delta \psi(x)}}^{F(x)} = 0 \]
Then this transformation is the symmetry of the transformation.\\
Now, if epsilon is function of x:
\[ \delta S = \displaystyle\int_{}^{} -J^\mu(x) \pdv{\epsilon(x)}{X^\mu}  \]
Integrating by parts,
\[ - \displaystyle\int_{}^{} \pdv{J^\mu \epsilon(x)}{X^\mu} + \pdv{J^\mu}{X^\mu} \epsilon(x) \]
Implies,
\[ \dv{J^\mu}{X^\mu} = 0 \]
\[ \partial_\mu J^\mu(x) = 0 \]
This is known as \textbf{Noether's Theorem}
\section{Lecture 5}
\subsection{Recap}
Now consider the Lagrangian density is also invariant:
Assuming(Global symmetry):
\[ \pdv{L}{\phi} F(x) + \pdv{L}{\partial_\mu \phi} \partial_\mu F(x) = 0 \]
\[ S = \displaystyle\int_{}^{} d^4 x L(\phi, \partial_\mu \phi) \]
take \( \phi \rightarrow(x) + \iota \epsilon(x) F(x) \)
\[ \delta S = \displaystyle\int_{}^{} \pdv{L}{\phi} \iota \epsilon F(X) + \pdv{L}{\partial_\mu \phi} \partial_\mu ( \iota \epsilon F(x)) \]
\[ = \displaystyle\int_{}^{} d^4 x \pdv{L}{\phi} \iota \epsilon F(x) 
+ \pdv{L}{\partial_\mu}  [ \iota \partial_\mu \epsilon(x) F(x) ] 
+ (\iota \epsilon(x) \partial_\mu F(x)) \]
\[ = \displaystyle\int_{}^{} d^4 x \pdv{L}{\partial_\mu \phi} \iota  F(x) \partial_\mu \epsilon(x)  \]
\[ J^\mu = - \pdv{L}{\partial_\mu \phi} \iota F(x) \]
\subsection{New}
\[ \phi(x) \rightarrow \phi(x + \epsilon(x)) \]
\[  = \phi(x) + \epsilon^\mu(x) \partial_\mu \phi \]
\[ \delta S = \displaystyle\int_{}^{} d^4 x [ \pdv{L}{\phi} \epsilon^\mu(x) \partial_\mu \phi + \pdv{L}{\partial_\mu \phi} \partial_\mu(\epsilon(x) \partial_\mu \phi(x))] \]
\[ = ... + \pdv{L}{\partial_\mu \phi} \epsilon^\nu(x) \partial_\mu \partial_\nu \phi(x) + \pdv{L}{\partial_\mu \phi} \partial_\mu \epsilon^\nu(x) \partial_\nu \phi(x) \]
\[ \pdv{L}{X^\mu} = \pdv{L}{\phi} \partial_\mu \phi + \pdv{L}{\partial_\nu \phi} \partial_\mu \partial_\nu \phi \]
Going back to previous eqn:
\[ = \displaystyle\int_{}^{} d^4 x (\epsilon^\nu(x) \partial_\nu L + \pdv{L}{\partial_\nu \phi} \partial_\mu \epsilon_\nu)(\partial_\nu \phi(x)) \]
Integrating by parts
\[ - \pdv{L}{\partial_\mu \phi} \epsilon^\nu \partial_\mu \partial_\nu \phi(x) + \pdv{L}{\partial_\mu \phi} \partial_\mu ( \epsilon^\nu `6_\nu \phi(x)) \]
\[ \delta S = \displaystyle\int_{}^{} d^4 x  \delta^\mu_\nu \partial_\mu \epsilon^\nu(x) L - \partial_\nu \epsilon^\nu(x) L + \pdv{L}{\partial_\mu \phi} \partial_\nu \phi(\partial_\mu \epsilon^\nu) \]
\[ = d^4 x [ (\partial_\mu \epsilon^\nu)[-\delta_\nu^\mu L + \pdv{L}{\partial_\mu \phi} \partial_\nu \phi]] \]
\[ \delta S = - \displaystyle\int_{}^{} d^4 x \partial_\mu \epsilon^\nu(x) T^\mu_\nu \]
Finally we get using \( \epsilon \rightarrow 0 at \infty \) and using equation of motion(Use the idea that for that equation of motion is derived from these kinds of variations) by applying by parts
\[ \partial_\mu T^\mu_\nu = 0 \]
Where:
\[ T^\mu_\nu = - \delta_\nu^\mu ... \]
This means energy momentum four Vector is conserved. \textbf{This happens ONLY because Lagrangian is NOT an explicit function of x and t.}. This is not related to Noether's Theorem.
\section{Lecture 5}
\[ S = \displaystyle\frac{1}{2} \displaystyle\int_{}^{} d^3 x dt [\dot{\phi}^2 - [\nabla \phi]^2 - m^2\phi^2] \]
See raghav Notes.
\[ \dot{\dot{\phi}} + \displaystyle\frac{R^2 + m^2}{\omega_k^2} \phi_k \]
See solution from raghav notes.
\[ [a_k^- , a_{k'}^+] = \delta(k - k') \]
TODO: Prove \( \dot{V_k} V_k^* - V_k \dot{V_k^*} = 2 \iota \) using phi form from raghav notes and above eqn\\
Above quantity is called Wrongskian
Choose \( V_K \)
\[ \dot{V_k} = \alpha_k \dot{U_k}(t) + \beta_k  U_k(t) \]
This will satisfy the diff eqn as long as 
\[ |\alpha_k|^2 - | \beta_k |^2 = 1 \]
This is known as Bogolyubov transformation\\
TODO: Why for above eqn.\\
Choice of mode function will not change \( \phi(x,t) \). It will just be in different "Basis"
\[ N = a^+ a^- \]
\( \ket{\partial^0_b} \) is a vaccum state(0 partice state) in b basis.
\[ \bra{\partial^0_b} N^(b) \ket{\partial^0_b} = 0  \]
Now we are interested in:
\[ \bra{\partial^0_b} N^(a) \ket{\partial^0_b} = |\beta_k|^2 \delta^3(0)  \]
which is non 0\\
Based on choice of mode function, one person may or may not see particles.\\
A particle is defined as a excitation with respect to a ground state.\\
In adiabatic approx time dependence is neglected\\
Choose a mode function:
\[ V_k^* = \displaystyle\frac{1}{\sqrt{\omega_k}} e^{- \iota \omega_k t}  \]
Show that \( \phi \) is Lorentz invariant after putting this mode function. Now the theory is Lorentz invariant!!
\section{Homework}
TODO: Homework
\[1.\   S = \displaystyle\int_{}^{} d^4 x [ - \displaystyle\frac{1}{4} F_{\mu\nu}F^{\mu\nu}] , F_{\mu\nu} = \partial_\mu A_\nu - \partial_\nu A_\mu \]
\begin{itemize}
	\item Find the EOM. Write the EOM in standard form identifying\( E^i = - F^d\) and \( \epsilon^{isk} B^k = -F^{ij} \)
	\item Compute \( T^{\mu\nu} \) . Notice that it is symmetric in \( \mu_{i\nu} \)
	\item Define \( T_{tilde}^{\mu\nu} =  T^{\mu\nu} ... \)
		Show that \(  T_{tilde}^{\mu\nu}\) is symmtric and satisfies \( \partial_\mu T_{tilde}^{\mu\nu} = 0 \)

\\
... \\
2. Consider the two complex Klein Gorden Scalar Fields
\[ L = \displaystyle\sum_{}^{} \]
\end{itemize}

\section{Lecture 8}
\[ L ~ (\partial_\mu \phi)^2 - m^2 \phi^2 \]
To include interaction(scattering) term.
\[  L ~ (\partial_\mu \phi)^2 - m^2 \phi^2 - \lambda \phi^3  \]
\( \phi^3  \) means one phi interacts with 2 more phis\\
\( \phi^2  \) means 2 phis just propagated\\
See diagram from photos.\\
COnstruct a scattering problem with two foch spaces. One for incpming particles, one for outgoing.
To calculate \( \bra{\beta, out} \ket{\alpha, in} \) we get the \( \alpha \beta  \) elements of the S matrix.
S is diagonal if there are no interactions.
Elements of S matrix are \( \bra{out} S \ket{in} \)
Similar to QM cross section:
\[ d \sigma = \displaystyle\frac{1}{2E_1 2E_2 |\nu_1 \nu_2 |} |S|^2 d\pi\] 
Where \( d \pi \) is the density of states(Final available states subject to energy momnetum conservation) which is given by
\[ \pi_j \displaystyle\frac{d^3 p_j}{(2\pi)^3 2 E_{p_j}} (2 \pi)^4 \delta^4 (p_1 + p_2 - p_3 - p_4) \]
Formula for S matrix LSZ reduction formula:
\[ \bra{p_3 p_4} S \ket{p_1 p_2} \]  =  Given in pics
\end{document}
