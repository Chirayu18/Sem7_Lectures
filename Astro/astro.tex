\documentclass{scrartcl}
\usepackage[utf8]{inputenc}
\usepackage[
backend=biber,
style=numeric,
sorting=ynt
]{biblatex}
\usepackage{amsmath}
\usepackage{physics}
\usepackage{graphicx}
\newtheorem{theorem}{Theorem}
\usepackage[a4paper, total={6in, 9in}]{geometry}
\addbibresource{references.bib}
\title{ Astronomy and astrophysics \\ \large{ PH } }

\author{Chirayu Gupta}
\date{}
\begin{document}
\section{Sixteenth August}
Some questions to ponder:
\subsection{Stars}
\begin{itemize}
	\item Source of their luminosity
	\item How big are they- radius/mass
	\item What are they made of?
	\item How long do they live
	\item How do they change over time
	\item What happens when they stop shining
\end{itemize}
\textbf{Luminosity}: About of light \textbf{received}
\subsection{Black Body radiation}
Plank's spectrum:
\[ B_\nu(\nu,T) = \displaystyle\frac{2h\nu^3}{c^2} \displaystyle\frac{1}{e^{h\nu / k_b T} -1 } \]
Leptons have conservation laws(Lepton conservation in HEP). No such law for photons.\\
Wien's displacement law has different peaks depending on the formula used for \( B\) i.e. whether you use \( B_\nu \) or \( B_\lambda \)\\
Colors observed are determined in terms of RGB. \textbf{Color} is the ratio of flux densities of two different wavelengths.
\[ S_\nu = B_\nu \cdot \Omega \] 
where \( \Omega \) is the solid angle\\
TODO:See HR diagrams\\
\subsection{Some relationships}
Luminosity \( \propto M^3 \) for \( M> 10 M_{sun} \)\\
\(\propto M^4 \) for \( M>0.5 M_{sun} \)
\subsection{Spectral Lines}
Provides information on:
\begin{itemize}
	\item Chemical composition
	\item Column density(Amount of substance in LOS)
	\item Gas temperature and pressure - Ratio of peaks gives temperature
	\item Velocity structure
	\item Ionization states
\end{itemize}
\end{document}
